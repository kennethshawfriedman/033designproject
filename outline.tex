\documentclass[10pt,journal,compsoc]{IEEEtran}

\hyphenation{op-tical net-works semi-conduc-tor}


\begin{document}
	
	\title{Lit Banjo; Keen Hoes:\\a 6.033 Design Project}
	
	
	\author{Obasi Onuoha, Kenneth Friedman, Joel Gustafson}% <-this % stops a space
	
	% The paper headers
	\markboth{We do 033 stuff}{}
	
	\IEEEtitleabstractindextext{%
		\begin{abstract}
			The outline goes here.
		\end{abstract}}
			
		% make the title area
		\maketitle
		
		\section{High Level System Overview}
		Our project increases net awesomeness by 2000\%.
		
		\subsection{Subsection Heading Here}
		Subsection text here.
		
		\subsubsection{Subsubsection Heading Here}
		Subsubsection text here.
		
		\section{Client}
		Are the machines the clients or are we?
		
		\section{AP}
		We have OVER 9000 APs.
		
		\section{Server}
		The majority of the computational complexity is handled by a central server, connected via reliable wired connection to all APs. The server aggregates network data and handles client assignment requests from APs. This section details the components of the central server and their interactions with the rest of the system.
		\subsection{Network State}
		The server maintains the state of traffic in network in a 6.006 data structure. Every 30 seconds, God rips the wings off of an angel, and Donald Trump sacrifices a campaign manager to Ba'al the Soul-Eater.
		
		\subsection{Client Assignment}
		The primary role of the sever is to respond to client assignment requests from APs with the MAC address of the best AP for the client to connect to. This routing function is computed by a neural net.
		
		\subsection{Network Analytics}
		In addition to client assignment routing, the server tracks network traffic statistics and reports aggregated data to IS\&T.
		
		\section{Communications Protocols}
		This section outlines the communications protocols used from the controller of a client to the AP, from the AP to the controller of a client, from the AP to the server, and from the server to the AP.
		
		\subsection{Client to AP}
		When a client connects to an AP, its controller immediately send a frame to the AP. The frame is of the following form:
		\begin{center}\textit{ src addr \textbar dst addr \textbar meta \textbar data  }\end{center}
		Where \textit{src addr} is the 48-bit MAC address of the client, \textit{dst addr} is the 48-bit MAC address of the AP to which the controller is communicating, and \textit{meta} is the 8-bit value 00000001. \textit{Data} is a variable-bit value defined as follows:
		\begin{center}\textit{R \textbar addrs}\end{center}
		Where \textit{R} is a 32-bit integer representing the maximum number of bits that the client will need to transmit over the course of any one second and \textit{addrs} is the value formed of the concatenation of the 48-bit MAC addresses of all of the APs within range of the client. The MAC addresses which compose \textit{addrs} are sorted in decreasing order of signal strength.\\
		\\
		Every 30 seconds, the client sends a message to its AP. This message is of the following form:
		\begin{center}\textit{ src addr \textbar dst addr \textbar meta \textbar data  }\end{center}
		Where \textit{src addr} is the 48-bit MAC address of the client, \textit{dst addr} is the 48-bit MAC address of the AP to which the controller is communicating, and \textit{meta} is the 8-bit value 00000100. \textit{Data} is a variable-bit value defined as follows:
		\begin{center}\textit{G \textbar A}\end{center}
		Where \textit{G} is a 32-bit integer representing the number of bits that the client generated over the past 30 seconds and \textit{A} is a 32-bit integer representing the number of bits that the client successfully sent over the past 30 seconds.
		
		\subsection{AP to Client}
		When an AP needs to tell a client to connect to a different AP within range of the client, it sends a frame of the following form:
		\begin{center}\textit{ src addr \textbar dst addr \textbar meta \textbar data  }\end{center}
		Where \textit{src addr} is the 48-bit MAC address of the AP sending the frame, \textit{dst addr} is the 48-bit MAC address of the client to which the AP wished to communicate, and \textit{meta} is the 8-bit value 00000010. \textit{Data} is 48-bit value specifying the AP to which the client should connect.\\
		\\
		When an AP needs to tell the user of a client to move physically in order to connect to a different AP which is not in the immediate range of the client, it sends a frame of the following form:
		\begin{center}\textit{ src addr \textbar dst addr \textbar meta \textbar data  }\end{center}
		Where \textit{src addr} is the 48-bit MAC address of the AP sending the frame, \textit{dst addr} is the 48-bit MAC address of the client to which the AP wished to communicate, and \textit{meta} is the 8-bit value 00000011. \textit{Data} is a 24-bit value defined as follows:
		\begin{center}\textit{bld \textbar rm}\end{center}
		Where \textit{bld} is a 12-bit binary integer specifying the building number of the desired AP and \textit{rm} is a 12-bit integer specifying the room number of the desired AP.
		
		\subsection{AP to Server}
		When a new client connects to an AP, it sends a message to the IS\&T server. This message is of the following form:
		\begin{center}\textit{maddr \textbar caddr \textbar R}\end{center}
		Where \textit{maddr} is the 48-bit MAC address of the AP, \textit{cddr} is the 48-bit MAC address of the client which just connected, and \textit{R} is a 32-bit integer specifying the maximum number of bits that the client will need to transmit over the course of any one second.\\
		\\
		Every 30 seconds, independent of any connected clients, the AP sends a message to the IS\&T server. This message is of the following form:
		\begin{center}\textit{maddr \textbar cnum \textbar rsum \textbar asum \textbar gsum}\end{center}
		Where \textit{maddr} is the 48-bit MAC address of the AP sending the message, \textit{cnum} is a 7-bit integer specifying number of clients connected to the AP sending the message, \textit{rsum} is a 39-bit integer specifying the maximum number of bits that the clients connected to the AP sending the message may need to send over any given second, \textit{asum} is a 20-bit integer specifying how many bits clients have transmitted to the AP sending the message over the last 30 seconds, and \textit{gsum} is a 39-bit integer specifying the number of bits that the clients connected to the AP sending the message have generated over the past 30 seconds. 
		
		\subsection{Server to AP}
		When the IS\&T server determines that an a client needs to connect to a different AP, it sends a message to the AP that client is currently connected to. This message takes the following form:
		\begin{center}\textit{caddr \textbar naddr \textbar rlct }\end{center}
		Where \textit{caddr} is the 48-bit MAC address of the client which is being directed to switch to a new AP and \textit{naddr} is the 48-bit MAC address of the AP to which the client is being directed to switch. \textit{rlct} is a 24-bit value composed of all 0's if the AP in question is in range of the client in question or a 24-bit value defined as follows if it is not:
		\begin{center}\textit{bld \textbar rm}\end{center}
		Where \textit{bld} is a 12-bit binary integer specifying the building number of the desired AP and \textit{rm} is a 12-bit integer specifying the room number of the desired AP.
		
		\appendices
		\section{Proof of the First Zonklar Equation}
		Appendix one text goes here.
		
		\section{}
		Appendix two text goes here.
		
\end{document}