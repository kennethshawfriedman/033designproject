\documentclass[10pt,journal,compsoc]{IEEEtran}

\hyphenation{op-tical net-works semi-conduc-tor}


\begin{document}
	
	\title{Lit Banjo; Keen Hoes:\\a 6.033 Design Project}
	
	
	\author{Obasi Onuoha, Kenneth Friedman, Joel Gustafson}% <-this % stops a space
	
	% The paper headers
	\markboth{We do 033 stuff}{}
	
	\IEEEtitleabstractindextext{%
		\begin{abstract}
			The outline goes here.
		\end{abstract}}
			
		% make the title area
		\maketitle
		
		\section{High Level System Overview}
		Our project increases net awesomeness by 2000\%.
		
		\subsection{Subsection Heading Here}
		Subsection text here.
		
		\subsubsection{Subsubsection Heading Here}
		Subsubsection text here.
		
		\section{Client}
		Are the machines the clients or are we?
		
		\section{AP}
		We have OVER 9000 APs.
		
		\section{Server}
		Does your server serve like our server?\\
		\\
		No. No it doesn't.
		
		\section{Communications Protocols}
		This section outlines the communications protocols used from the controller of a client to the AP, from the AP to the controller of a client , from the AP to the server, and from the server to the AP.
		
		\subsection{Client to AP}
		When a client connects to an AP, its controller immediately send a frame to the AP. The frame is of the following form:
		\begin{center}\textit{ src addr \textbar dst addr \textbar meta \textbar data  }\end{center}
		Where \textit{src addr} is the 48-bit MAC address of the client, \textit{dst addr} is the 48-bit MAC address of the AP to which the controller is communicating, and \textit{meta} is the 8-bit value 00000001. \textit{Data} is an variable-bit value defined as follows:
		\begin{center}\textit{R \textbar addrs}\end{center}
		Where \textit{R} is a 32-bit integer representing the maximum number of bits that the client will need to transmit over the course of any one second and \textit{addrs} is the value formed of the concatenation of the 48-bit MAC addresses of all of the APs within range of the client. The MAC addresses which compose \textit{addrs} are sorted in decreasing order of signal strength.
		
		\appendices
		\section{Proof of the First Zonklar Equation}
		Appendix one text goes here.
		
		\section{}
		Appendix two text goes here.
		
\end{document}